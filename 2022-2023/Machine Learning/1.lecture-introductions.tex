% Created 2022-08-04 jeu. 14:38
% Intended LaTeX compiler: pdflatex
\documentclass[10pt]{beamer}
\usepackage[utf8]{inputenc}
\usepackage[T1]{fontenc}
\usepackage{graphicx}
\usepackage{longtable}
\usepackage{wrapfig}
\usepackage{rotating}
\usepackage[normalem]{ulem}
\usepackage{amsmath}
\usepackage{amssymb}
\usepackage{capt-of}
\usepackage{hyperref}
\usepackage{minted}
\usetheme{Berkeley}
\definecolor{UBCblue}{rgb}{0.54706, 0.13725, 0.26667} % UBC Blue (primary)
\usecolortheme[named=UBCblue]{structure}
\setlength{\parskip}{5pt}
\newcommand{\footnoteframe}[1]{\footnote[frame]{#1}}
\addtobeamertemplate{footnote}{}{\vspace{2ex}}
\usepackage{xcolor}
\definecolor{LightGray}{gray}{0.95}
\usetheme{default}
\author{Jay Morgan}
\date{<TODO>}
\title{Machine Learning}
\subtitle{Lecture 1 - Introductions}
\hypersetup{
 pdfauthor={Jay Morgan},
 pdftitle={Machine Learning},
 pdfkeywords={},
 pdfsubject={},
 pdfcreator={Emacs 28.1 (Org mode 9.5.4)}, 
 pdflang={English}}
\begin{document}

\maketitle
\begin{frame}{Outline}
\tableofcontents
\end{frame}


\section*{Introductions}
\label{sec:org025e022}

Welcome to all the new students! Here I am going to be talking about Machine Learning
and all of the great things that this "technology" has to offer. To begin our course,
I shall start with a bit of house keeping -- more specifically, I will be talking
about what exactly we'll be learning about in the course (Machine Learning is a broad
subject after-all). In addition, I will tell you where you can find the resources
related to the course and how you can contact me, should you have any questions.

\begin{frame}[label={sec:org5b47f76}]{What this course is about?}
In this course, we will be learning about Machine Learning: firstly, what Machine
Learning actually is; secondly, we'll take a look at some of the algorithms within
the scope of Machine Learning, and develop an intuition about how these algorithms
work and when they would be useful; and finally, how we can compare and evaluate the
algorithms we've learnt about.
\end{frame}

\begin{frame}[label={sec:org74e7c3d},fragile]{How this course will be taught}
 I intended to deliver this course via a series of lectures. These lectures will be
accompanied by the PDF lecture slides, in which I will provide the definitions and
provide reference links should you wish to do some extra reading.

In some situations, I would also like to supplement my algorithmic definitions with
some programming code -- for this I will use the \href{https://julialang.org/}{Julia} programming language. The code
snippets would look something like:

\begin{minted}[frame=lines,linenos=true,firstnumber=last,fontsize=\footnotesize,bgcolor=LightGray,xleftmargin=5pt,tabsize=2,breaklines=true,numbersep=10pt]{julia}
x = Float32.([1, 2, 3, 4]);
y = x .+ randn(length(x))
\end{minted}

\begin{verbatim}
4-element Vector{Float64}:
 1.8817576050704024
 2.051062476034138
 4.340649570113399
 3.1284623590241476
\end{verbatim}
\end{frame}

\begin{frame}[label={sec:org6e38224}]{Outline of the course}
\begin{center}
\begin{tabular}{rlll}
Lecture & Expected date & Length & Topic\\
\hline
1 & NOW & 2 hours & Introduction\\
 &  &  & \\
\end{tabular}
\end{center}
\end{frame}

\begin{frame}[label={sec:org02cc03a}]{About Me}
My name is Dr Jay Morgan. I am a researcher at the Université de Toulon, where I am
developing Deep Learning models (a sub-field of Machine Learning research) for the
study of astrophysical phenomenon.

You can find more information and links on my personal (LIS-Lab) website:
\url{https://pageperso.lis-lab.fr/jay.morgan/}

I also publish libraries and source code online:
\begin{itemize}
\item Github: \url{https://github.com/jaypmorgan}
\item Gitlab: \url{https://gitlab.com/jaymorgan}
\item Source Hut: \url{https://sr.ht/\~jaymorgan/}
\end{itemize}

If you have any questions, you can email me at jay.morgan@univ-tln.fr
\end{frame}

\begin{frame}[label={sec:orgbdeaf2f}]{Where you can find the resources}
I try to make this course as accessible as possible, which means that I host these
slides in a variety of ways to suit you.

Firstly, you can find the links to all my courses on my personal website at:
\url{https://pageperso.lis-lab.fr/jay.morgan/teaching.html}

Here you can find the links to each lecture in a PDF or HTML format. Additionally,
you can view the source code used to make these lectures on source hut:
\url{https://git.sr.ht/\~jaymorgan/teaching}. On this git repository you can find all my
lectures from all years.
\end{frame}

\section*{What is learning, anyway?}
\label{sec:org71ca252}

We'll begin our journey into the world of Machine Learning by tackling the question
of what it means to 'learn' -- how may a machine actually \emph{learn} anything?

\begin{frame}[label={sec:orga6b898e}]{Study of Mice}
To begin to answer the question of learning, we may turn to nature for
advice. Principally, if we look at the studies conducted with Mice we find some idea
to notion of learning.
\end{frame}

\begin{frame}[label={sec:org3386874}]{Computer Programs}
For a more formal definition of how computer programs could be said to learn, we have:

\begin{quote}
A computer program is said to learn from experience \(E\) with respect
to some class of tasks \(T\) and performance measure \(P\), if its performance
a tasks in \(T\), as measured by \(P\), improves with experience \(E\).
\end{quote}

(Mitchell, Tom M, 1997)
\end{frame}

\section*{Bibliography}
\label{sec:orga166c57}

\noindent
Mitchell, Tom M (1997). \emph{Machine learning}, McGraw-hill New York.
\end{document}
