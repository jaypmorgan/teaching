% Created 2022-08-19 ven. 10:45
% Intended LaTeX compiler: pdflatex
\documentclass[10pt]{beamer}
\usepackage[utf8]{inputenc}
\usepackage[T1]{fontenc}
\usepackage{graphicx}
\usepackage{longtable}
\usepackage{wrapfig}
\usepackage{rotating}
\usepackage[normalem]{ulem}
\usepackage{amsmath}
\usepackage{amssymb}
\usepackage{capt-of}
\usepackage{hyperref}
\usepackage{minted}
\usepackage{xcolor}
\definecolor{LightGray}{gray}{0.95}
\usefonttheme{professionalfonts}
\setlength{\parskip}{5pt}
\newcommand{\footnoteframe}[1]{\footnote[frame]{#1}}
\addtobeamertemplate{footnote}{}{\vspace{2ex}}
\usepackage{tabularx}
\usepackage{booktabs}
\DefineVerbatimEnvironment{verbatim}{Verbatim}{fontsize=\scriptsize}
\usetheme{Berkeley}
\author{Jay Morgan}
\date{2021-10-01}
\title{Programming Level-up}
\subtitle{Lecture 2 - More advanced Python \& Classes}
\hypersetup{
 pdfauthor={Jay Morgan},
 pdftitle={Programming Level-up},
 pdfkeywords={},
 pdfsubject={},
 pdfcreator={Emacs 27.1 (Org mode 9.4.6)}, 
 pdflang={English}}
\begin{document}

\maketitle
\begin{frame}{Outline}
\tableofcontents
\end{frame}


\section{Proxy}
\label{sec:orgbe8cf79}

\subsection{Univ-tln proxy}
\label{sec:orgd1f7480}

\begin{frame}[label={sec:org2001cc4},fragile]{Setting up a proxy in Linux -- environment variables}
 Environment variables are variables that are set in the Linux environment and are
used to configure some high-level details in Linux.

The command to create/set an environment is:

\begin{verbatim}
export VARIABLE_NAME=''
\end{verbatim}

Exporting a variable in this way will mean \texttt{VARIABLE\_NAME} will be accessible while
you're logged in. Every time you log in you will have to set this variable again.
\end{frame}

\begin{frame}[label={sec:org21eb05a},fragile]{Setting up a proxy in Linux -- univ-tln specific}
 In the université de Toulon, you're required to use the university's proxy server to
access the internet. Therefore, in Linux at least, you will have to tell the system
where the proxy server is with an environment variable.

\begin{minted}[frame=lines,linenos=true,firstnumber=last,fontsize=\footnotesize,bgcolor=LightGray,xleftmargin=5pt,tabsize=2,breaklines=true,numbersep=10pt]{bash}
export HTTP_PROXY='<username>:<password>@proxy.univ-tln.fr:3128'
export HTTPS_PROXY='<username>:<password>@proxy.univ-tln.fr:3128'
export FTP_PROXY='<username>:<password>@proxy.univ-tln.fr:3128'
\end{minted}

\alert{NOTE}: Watch out for special characters in your password! They will have to be URL encoded.
\end{frame}

\begin{frame}[label={sec:orge0b1395},fragile]{Setting up a proxy in the .bashrc}
 If you don't wish to set the variable every time log in, you should enter the same
commands into a \texttt{.bashrc} in your home directory.

\begin{minted}[frame=lines,linenos=true,firstnumber=last,fontsize=\footnotesize,bgcolor=LightGray,xleftmargin=5pt,tabsize=2,breaklines=true,numbersep=10pt]{bash}
export HTTP_PROXY='...'
export HTTPS_PROXY='...'
export FTP_PROXY='...'
\end{minted}

When you log in, the \texttt{.bashrc} file will be run and these variables will be set for you.
\end{frame}

\section{Dealing with Errors}
\label{sec:org2ef4256}

\subsection{Exceptions}
\label{sec:org1aa8008}

\begin{frame}[label={sec:org7902498},fragile]{Dealing with Errors}
 When programming, its good to be defensive and handle errors gracefully. For example,
if you're creating a program, that as part of its process, reads from a file, its
possible that this file may not exist at the point the program tries to read it. If
it doesn't exist, the program will crash giving an error such as: \texttt{FileNotfoundError}.

Perhaps this file is non-essential to the operation of the program, and we can
continue without the file. In these cases, we will want to appropriately catch the
error to prevent it from stopping Python.
\end{frame}

\begin{frame}[label={sec:org40413a4},fragile]{Try-catch}
 Try-catches are keywords that introduce a scope where the statements are executed,
and if an error (of a certain type IndexError in this example) occurs, different
statements could be executed.

In this example, we are trying to access an element in a list using an index larger
than the length of the list. This will produce an \texttt{IndexError}. Instead of exiting
Python with an error, however, we can catch the error, and print a string.

\begin{minted}[frame=lines,linenos=true,firstnumber=last,fontsize=\footnotesize,bgcolor=LightGray,xleftmargin=5pt,tabsize=2,breaklines=true,numbersep=10pt]{python}
x = [1, 2, 3]

try:
    print(x[3])
except IndexError:
    print("Couldn't access element")
\end{minted}

\begin{verbatim}
Results: 
# => Couldn't access element
\end{verbatim}
\end{frame}

\begin{frame}[label={sec:orgbc5b2b4},fragile]{Try-catch -- capturing messages}
 If we wanted to include the original error message in the print statement, we can use
the form:

\begin{verbatim}
except <error> as <variable>
\end{verbatim}

This provides us with an variable containing the original error that we can use later
on in the try-catch form.

\begin{minted}[frame=lines,linenos=true,firstnumber=last,fontsize=\footnotesize,bgcolor=LightGray,xleftmargin=5pt,tabsize=2,breaklines=true,numbersep=10pt]{python}
x = [1, 2, 3]

try:
    print(x[3])
except IndexError as e:
    print(f"Couldn't access elements at index beacuse: {e}")
\end{minted}

\begin{verbatim}
Results: 
# => Couldn't access elements at index beacuse: list index out of range
\end{verbatim}
\end{frame}

\begin{frame}[label={sec:org45aa20b}]{Types of exceptions}
There are numerous types of errors that could occur in a Python. Here are just some
of the most common.

\begin{itemize}
\item IndexError -- Raised when a sequence subscript is out of range.
\item ValueError -- Raised when an operation or function receives an argument that has the right type but an inappropriate value
\item AssertionError -- Raised when an assert statement fails.
\item FileNotFoundError -- Raised when a file or directory is requested but doesn’t exist.
\end{itemize}

The full list of exceptions in Python 3 can be found at: \url{https://docs.python.org/3/library/exceptions.html}
\end{frame}

\begin{frame}[label={sec:org0152903},fragile]{Assertions}
 One of the previous errors (\texttt{AssertionError}) occurs when an assert statement
fails. Assert is a keyword provided to test some condition and raise an error if the
condition is false. It typically requires less code than an \texttt{if}-statement that raises
an error, so they might be useful for checking the inputs to functions, for example:

\begin{minted}[frame=lines,linenos=true,firstnumber=last,fontsize=\footnotesize,bgcolor=LightGray,xleftmargin=5pt,tabsize=2,breaklines=true,numbersep=10pt]{python}
def my_divide(a, b):
    assert b != 0
    return a / b

my_divide(1, 2)
my_divide(1, 0)
\end{minted}

Here we are checking that the divisor is not a 0, in which case division is not defined.
\end{frame}

\section{OOP}
\label{sec:org6204ecc}
\subsection{Classes}
\label{sec:org4a7ad4d}

\begin{frame}[label={sec:orgc949d8d}]{Introduction to classes}
A class is some representation (can be abstract) of an object. Classes can be used to
create some kind of structure that can be manipulated and changed, just like the ways
you've seen with lists, dictionaries, etc.

Classes allow us to perform Object-oriented Programming (OOP), where we represent
concepts by classes.

But to properly understand how classes work, and why we would want to use them, we
should take a look at some examples.
\end{frame}

\begin{frame}[label={sec:orgfb12de4},fragile]{Basic syntax}
 We're going to start off with the very basic syntax, and build up some more complex
classes.

To create a class, we use the \texttt{class} keyword, and give our new class a name. This
introduces a new scope in Python, the scope of the class.

Typically, the first thing we shall see in the class is the \texttt{\_\_init\_\_} function.

\begin{minted}[frame=lines,linenos=true,firstnumber=last,fontsize=\footnotesize,bgcolor=LightGray,xleftmargin=5pt,tabsize=2,breaklines=true,numbersep=10pt]{python}
class <name_of_class>:
    def __init__(self, args*):
        <body>
\end{minted}
\end{frame}

\begin{frame}[label={sec:orgc661b94},fragile]{Init method}
 The \texttt{\_\_init\_\_} function is a function that gets called automatically as soon as a class
is made. This init function can take many arguments, but must always start with a
\texttt{self}.

In this example, we are creating a class that represents an x, y coordinate. We've
called this class \texttt{Coordinate}, and we've defined our init function to take an x and y
values when the class is being created.

\alert{Note} its more typical to use titlecase when specifying the class name. So when
reading code its easy to see when you're creating a class versus calling a
function. You should use this style.

\begin{minted}[frame=lines,linenos=true,firstnumber=last,fontsize=\footnotesize,bgcolor=LightGray,xleftmargin=5pt,tabsize=2,breaklines=true,numbersep=10pt]{python}
class Coordinate:
    def __init__(self, x, y):
        self.x = x
        self.y = y
\end{minted}
\end{frame}

\begin{frame}[label={sec:orgd59f369},fragile]{Instantiating}
 To create an \emph{instance} of this class, call the name of the class as you would a
function, and pass any parameters you've defined in the init function.

In this example, we are creating a new vector using \texttt{Vector(...)} and we're passing the
x and y coordinate.

\begin{minted}[frame=lines,linenos=true,firstnumber=last,fontsize=\footnotesize,bgcolor=LightGray,xleftmargin=5pt,tabsize=2,breaklines=true,numbersep=10pt]{python}
class Vector:
    def __init__(self, x, y):
        self.x = x
        self.y = y


point_1 = Vector(5, 2)
\end{minted}
\end{frame}

\begin{frame}[label={sec:orgb3ce4cd},fragile]{Class variables}
 In the previous example, we've been creating a class variables by using
\texttt{self.<variable\_name>}. This is telling Python \emph{this class should have a variable of
this name}.

It allows then to reference the variable when working with the class.

\begin{minted}[frame=lines,linenos=true,firstnumber=last,fontsize=\footnotesize,bgcolor=LightGray,xleftmargin=5pt,tabsize=2,breaklines=true,numbersep=10pt]{python}
class Vector:
    def __init__(self, x, y):
        self.x = x
        self.y = y
        self.length = self.x + self.y

point_1 = Vector(5, 2)
print(point_1.x)
print(point_1.y)
print(point_1.length)
\end{minted}

\begin{verbatim}
Results: 
# => 5
# => 2
# => 7
\end{verbatim}
\end{frame}

\begin{frame}[label={sec:org06a0395},fragile]{Class Methods}
 A class can have many methods associated with it. To create a new method, we create a
function within the scope of the class, remember that the first parameter of the
function should be \texttt{self}.

Even in these functions, we can refer to our \texttt{self.x} and \texttt{self.y} within this new
function.

You'll notice that to call this function, we using the \texttt{.length()} method similar to
how we've worked with strings/lists/etc. This is because in Python, everything is an
object!

\begin{minted}[frame=lines,linenos=true,firstnumber=last,fontsize=\footnotesize,bgcolor=LightGray,xleftmargin=5pt,tabsize=2,breaklines=true,numbersep=10pt]{python}
class Vector:
    def __init__(self, x, y):
        self.x = x
        self.y = y

    def length(self):
        return self.x + self.y


my_point = Vector(2, 5)
print(my_point.length())
\end{minted}

\begin{verbatim}
Results: 
# => 7
\end{verbatim}
\end{frame}

\begin{frame}[label={sec:org9b73e65},fragile]{dunder-methods}
 While we could, for example, create a function called \texttt{.print()}, sometimes we would
like to use the in built functions like \texttt{print()}. When creating a class, there is a
set of \emph{dunder-methods} (double-under to reference the two '\texttt{\_\_}' characters either side
of the function name).

One of these dunder-methods is \texttt{\_\_repr\_\_}, which allows us to specify how the object
looks when its printed.
\end{frame}

\begin{frame}[label={sec:org1a75e57},fragile]{dunder-methods}
 \begin{minted}[frame=lines,linenos=true,firstnumber=last,fontsize=\footnotesize,bgcolor=LightGray,xleftmargin=5pt,tabsize=2,breaklines=true,numbersep=10pt]{python}
class OldVector:
    def __init__(self, x, y):
        self.x = x
        self.y = y

print(OldVector(2, 5))

class Vector:
    def __init__(self, x, y):
        self.x = x
        self.y = y

    def __repr__(self):
        return f"Vector({self.x}, {self.y})"

print(Vector(2, 5))
\end{minted}

\begin{verbatim}
Results: 
# => <__main__.OldVector object at 0x7f658721e250>
# => Vector(2, 5)
\end{verbatim}
\end{frame}

\begin{frame}[label={sec:org8ad744a},fragile]{dunder-methods}
 There are many more dunder-methods you should know when creating classes. We shall go through:

\begin{itemize}
\item \texttt{\_\_len\_\_} -- specify how the length of the class should be computed.
\item \texttt{\_\_getitem\_\_} -- how to index over the class
\item \texttt{\_\_call\_\_} -- how to use the class like a function
\item \texttt{\_\_iter\_\_} -- what to do when iteration starts
\item \texttt{\_\_next\_\_} -- what to do at the next step of the iteration
\end{itemize}
\end{frame}

\begin{frame}[label={sec:org4717e23},fragile]{\texttt{\_\_len\_\_}}
 The \texttt{\_\_len\_\_} function allows us to specify how the \texttt{len()} function acts on the
class. Take this hypothetical dataset. We create a \texttt{\_\_len\_\_} function that returns the
length of the unique elements in the dataset.

\begin{minted}[frame=lines,linenos=true,firstnumber=last,fontsize=\footnotesize,bgcolor=LightGray,xleftmargin=5pt,tabsize=2,breaklines=true,numbersep=10pt]{python}
class Dataset:
    def __init__(self, data):
        self.data = data

    def __len__(self):
        """Return the length of unique elements"""
        return len(set(self.data))

data = Dataset([1, 2, 3, 3, 3, 5, 1])
print(len(data))
\end{minted}

\begin{verbatim}
Results: 
# => 4
\end{verbatim}
\end{frame}

\begin{frame}[label={sec:orgaaf0fee},fragile]{\texttt{\_\_getitem\_\_}}
 Next \texttt{\_\_getitem\_\_} allows us to index over a class. This new function must include \texttt{self}
and a variable to pass the index. Here I've used \texttt{idx}. In this function I am simply
indexing on the on the classes \texttt{self.data}.

\begin{minted}[frame=lines,linenos=true,firstnumber=last,fontsize=\footnotesize,bgcolor=LightGray,xleftmargin=5pt,tabsize=2,breaklines=true,numbersep=10pt]{python}
class Dataset:
    def __init__(self, data):
        self.data = data

    def __getitem__(self, idx):
        return self.data[idx]

data = Dataset([1, 2, 3, 3, 3, 5, 1])
print(data[2])
\end{minted}

\begin{verbatim}
Results: 
# => 3
\end{verbatim}
\end{frame}

\begin{frame}[label={sec:org8952eda},fragile]{\texttt{\_\_call\_\_}}
 In a small number of cases, it is nice to use the class just like a function. This is
what \texttt{\_\_call\_\_} allows us to do. In this function we specify what should happen when
class is 'called' like a function. In this simple example, we are creating a function
that prints the type of food being used as a parameter to the function.

\begin{minted}[frame=lines,linenos=true,firstnumber=last,fontsize=\footnotesize,bgcolor=LightGray,xleftmargin=5pt,tabsize=2,breaklines=true,numbersep=10pt]{python}
class Jaguar:
    def __call__(self, food):
        print(f"The jaguar eats the {food}.")

food = "apple"
animal = Jaguar()

animal(food)
\end{minted}

\begin{verbatim}
Results: 
# => The jaguar eats the apple.
\end{verbatim}
\end{frame}

\begin{frame}[label={sec:orgfb0bbd5},fragile]{\texttt{\_\_iter\_\_} and \texttt{\_\_next\_\_}}
 \texttt{\_\_iter\_\_} and \texttt{\_\_next\_\_} allow us to make our class iterable, i.e. we can use it in a
\texttt{for} loop for example.

The \texttt{\_\_iter\_\_} function should define what happens when we start the iteration, and
\texttt{\_\_next\_\_} defines what happens at every step of the iteration.

Let's take a look at an example where we have an iterable set of prime numbers.
\end{frame}

\begin{frame}[label={sec:orgd3229f8},fragile]{\texttt{\_\_iter\_\_} and \texttt{\_\_next\_\_}}
 \begin{minted}[frame=lines,linenos=true,firstnumber=last,fontsize=\footnotesize,bgcolor=LightGray,xleftmargin=5pt,tabsize=2,breaklines=true,numbersep=10pt]{python}
class Primes:
    def __init__(self):
        self.primes = [2, 3, 5, 7, 11]

    def __iter__(self):
        self.idx = 0
        return self

    def __len__(self):
        return len(self.primes)

    def __next__(self):
        if self.idx < len(self):
            item = self.primes[self.idx]
            self.idx += 1
            return item
        else:
            raise StopIteration
\end{minted}
\end{frame}

\begin{frame}[label={sec:org027467f},fragile]{\texttt{\_\_iter\_\_} and \texttt{\_\_next\_\_}}
 And now we can iterate over this class

\begin{minted}[frame=lines,linenos=true,firstnumber=last,fontsize=\footnotesize,bgcolor=LightGray,xleftmargin=5pt,tabsize=2,breaklines=true,numbersep=10pt]{python}
prime_numbers = Primes()

for prime_number in prime_numbers:
    print(prime_number)
\end{minted}

\begin{verbatim}
Results: 
# => 2
# => 3
# => 5
# => 7
# => 11
\end{verbatim}
\end{frame}

\begin{frame}[label={sec:orgcfec86b},fragile]{Inheritance}
 One special thing about OOP is that its normally designed to provide inheritance --
this is true in Python. Inheritance is where you have a base class, and other classes
inherit from this base class. This means that the class that inherits from the base
class has access to the same methods and class variables. In some cases, it can
override some of these features.

Let's take a look an example.

\begin{minted}[frame=lines,linenos=true,firstnumber=last,fontsize=\footnotesize,bgcolor=LightGray,xleftmargin=5pt,tabsize=2,breaklines=true,numbersep=10pt]{python}
class Animal:
    def growl(self):
        print("The animal growls")

    def walk(self):
        raise NotImplementError
\end{minted}

Here we have created a simple class called Animal, that has two functions, one of
which will raise an error if its called.
\end{frame}

\begin{frame}[label={sec:org5b365ec},fragile]{Inheritance}
 We can inherit from this Animal class by placing our base class in \texttt{()} after the new
class name.

Here we are creating two classes, Tiger and Duck. Both of these new classes inherit
from Animal. Also, both of these classes are overriding the walk functions. But they
are not creating a growl method themselves.

\begin{minted}[frame=lines,linenos=true,firstnumber=last,fontsize=\footnotesize,bgcolor=LightGray,xleftmargin=5pt,tabsize=2,breaklines=true,numbersep=10pt]{python}
class Tiger(Animal):
    def walk(self):
        print("The Tiger walks through the jungle")

class Duck(Animal):
    def walk(self):
        print("The Duck walks through the jungle")
\end{minted}
\end{frame}

\begin{frame}[label={sec:orgc2a9be8},fragile]{Inheritance}
 Look at what happens when we create instances of these classes, and call the
functions. First we see that the correct method has been called. I.e. for the duck
class, the correct \texttt{walk} method was called.

\begin{minted}[frame=lines,linenos=true,firstnumber=last,fontsize=\footnotesize,bgcolor=LightGray,xleftmargin=5pt,tabsize=2,breaklines=true,numbersep=10pt]{python}
first_animal = Tiger()
second_animal = Duck()

first_animal.walk()
second_animal.walk()
\end{minted}

\begin{verbatim}
Results: 
# => The Tiger walks through the jungle
# => The Duck walks through the jungle
\end{verbatim}
\end{frame}

\begin{frame}[label={sec:orgf21c525},fragile]{Inheritance}
 But what happens if we call the \texttt{.growl()} method?

\begin{minted}[frame=lines,linenos=true,firstnumber=last,fontsize=\footnotesize,bgcolor=LightGray,xleftmargin=5pt,tabsize=2,breaklines=true,numbersep=10pt]{python}
first_animal.growl()
second_animal.growl()
\end{minted}

\begin{verbatim}
Results: 
# => The animal growls
# => The animal growls
\end{verbatim}


We see that it still works. Even though both Duck and Tiger didn't create a \texttt{.growl()}
method, it inherited it from the base class Animal. This works for class methods and
class variables.
\end{frame}

\section{Exercise}
\label{sec:org7322f9b}

\subsection{Exercise}
\label{sec:org58c114d}

\begin{frame}[label={sec:org522b7cb},fragile]{An object based library system}
 We're going to improve on our library system from last lecture. Instead of a
\texttt{functional} style of code, we're going to use a OOP paradigm to create our solution.

Like last time, we're going to create our solution one step at a time.

First, we need to create our class called \texttt{Database}. This database is going to take an
optional parameter in its init function -- the data. If the user specifies data
(represented as a list of dictionaries like last time), then the class will populate
a class variable called data, else this class variable will be set to an empty list.

Summary:
\begin{itemize}
\item Create a class called \texttt{Database}.
\item When creating an instance of \texttt{Database}, the user can optionally specify a list of
dictionaries to initialise the class variable \texttt{data} with. If no data is provided,
this class variable will be initialised to an empty list.
\end{itemize}
\end{frame}

\begin{frame}[label={sec:org885e330},fragile]{Adding data}
 We will want to include a function to add data to our database.

Create a class method called \texttt{add}, that takes three arguments (in addition to \texttt{self} of
course), the title, the author, and the release date.

This add function adds the new book entry to the end of \texttt{data}. Populate this database
with the following information.

\begin{center}
\begin{tabularx}{\textwidth}{XXX}
\toprule
Title & Author & Release Date\\
\midrule
Moby Dick & Herman Melville & 1851\\
A Study in Scarlet & Sir Arthur Conan Doyle & 1887\\
Frankenstein & Mary Shelley & 1818\\
Hitchhikers Guide to the Galaxy & Douglas Adams & 1879\\
\bottomrule
\end{tabularx}
\end{center}
\end{frame}

\begin{frame}[label={sec:org77fb967},fragile]{Locating a book}
 Create a class method called locate by tile that takes the title of the book to look
up, and returns the dictionary of all books that have this title. Unlike last time,
we don't need to pass the \texttt{data} as an argument, as its contained within the class.
\end{frame}

\begin{frame}[label={sec:org181149b},fragile]{Updating our database}
 Create a class method called \texttt{update} that takes 4 arguments:, 1) the key of the value
we want to update 2) the value we want to update it to 3) the key we want to check
to find out if we have the correct book and 4) the value of the key to check if we
have the correct book.

\begin{minted}[frame=lines,linenos=true,firstnumber=last,fontsize=\footnotesize,bgcolor=LightGray,xleftmargin=5pt,tabsize=2,breaklines=true,numbersep=10pt]{python}
db.update(key="release year", value=1979, where_key="title",
          where_value="Hitchhikers Guide to the Galaxy")
\end{minted}

Use this to fix the release data of the Hitchhiker's book.
\end{frame}

\begin{frame}[label={sec:org741e82a},fragile]{Printed representation}
 Using the \texttt{\_\_str\_\_} dunder-method (this is similar to \texttt{\_\_repr\_\_} as we saw before),
create a function that prints out a formatted representation of the entire database
as a string. Some of the output should look like:

\begin{minted}[frame=lines,linenos=true,firstnumber=last,fontsize=\footnotesize,bgcolor=LightGray,xleftmargin=5pt,tabsize=2,breaklines=true,numbersep=10pt]{python}
Library System
--------------

Entry 1:
- Name: Moby Dick
- Author: Herman Melville
- Release Date: 1851
...
\end{minted}
\end{frame}

\begin{frame}[label={sec:org721f50d},fragile]{Extending our OOP usage}
 So far we've used a list of dictionaries. One issue with this is that there is no
constraints on the keys we can use. This will certainly create problems if certain
keys are missing.

Instead of using dictionaries. We can create another class called \texttt{Book} that will take
three arguments when it is initialised: \texttt{name}, \texttt{author}, and \texttt{release\_date}. The init
function should initialise three class variables to save this information.

Modify the database to, instead of working with a list of dictionaries, work with a
list of Book objects.
\end{frame}

\begin{frame}[label={sec:org3fc7ef3},fragile]{Printed representation -- challenge.}
 Improve upon the printed representation of the last exercise but instead of bulleted
lists, use formatted tables using \texttt{f-string} formatting
(\url{https://zetcode.com/python/fstring/}).

The output should look like this:

\begin{minted}[frame=lines,linenos=true,firstnumber=last,fontsize=\footnotesize,bgcolor=LightGray,xleftmargin=5pt,tabsize=2,breaklines=true,numbersep=10pt]{python}
Library System
--------------

| Name           | Author           | Release Data |
|----------------|------------------|--------------|
| Moby Dick      | Herman Melville  |         1851 |
...
\end{minted}

Notice how Release date is right justified, while Name and Author are left justified.
\end{frame}
\end{document}
