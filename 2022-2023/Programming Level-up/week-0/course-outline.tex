% Created 2021-09-11 sam. 16:10
% Intended LaTeX compiler: pdflatex
\documentclass[11pt]{article}
\usepackage[utf8]{inputenc}
\usepackage[T1]{fontenc}
\usepackage{graphicx}
\usepackage{grffile}
\usepackage{longtable}
\usepackage{wrapfig}
\usepackage{rotating}
\usepackage[normalem]{ulem}
\usepackage{amsmath}
\usepackage{textcomp}
\usepackage{amssymb}
\usepackage{capt-of}
\usepackage{hyperref}
\usepackage{minted}
\author{Jay Morgan}
\date{September 2021}
\title{Programming Level-up Course - Overview}
\hypersetup{
 pdfauthor={Jay Morgan},
 pdftitle={Programming Level-up Course - Overview},
 pdfkeywords={},
 pdfsubject={},
 pdfcreator={Emacs 27.1 (Org mode 9.4.6)}, 
 pdflang={English}}
\begin{document}

\maketitle
\tableofcontents


\section{Welcome!}
\label{sec:org0c189a5}

Welcome to the \textbf{Programming Level-up Course}. In this series of lectures, we will cover
everything we need to be able to program in a Linux-based environment, and use the
high performance computers (also called cluster/supercomputers) to run experiments. 

\section{Contact information}
\label{sec:orgfe93384}

You can find my personal page over at: \url{https://pageperso.lis-lab.fr/jay.morgan/}

As we progress through the lectures, I will also make the course publicly available. These lectures will be hosted at:
\url{https://pageperso.lis-lab.fr/jay.morgan/teaching.html} in a variety of formats
(i.e. PDF, HTML).

If you have any questions please email me directly. My email address is
\texttt{jay.morgan@lis-lab.fr}. Other modes of contact can be found on my personal website
listed above.

\section{Delivery}
\label{sec:orgf05bca2}
\begin{itemize}
\item 10 2-hour sessions
\item Each session will be conducted in a computer lab and will be combination of a
lecture and exercises
\item The course is intended to get everyone up-to-speed with the various facets of
programming you will need in other courses.
\item If time allows, each lecture will have an exercise -- though this is not a strict
requirement. These exercises are not marked but serve to get you thinking about
what you've learnt in the lecture.
\end{itemize}

\section{Resources}
\label{sec:org70f37a3}

This course aims to deliver everything you need. If you attend each lecture, you will
know what you need for the following lectures. Despite this design, however, I have
included a list of additional resources below. These resources are optional, but they
will take you beyond what you're taught in these sessions and enable you to become a
Programming Master!

\subsection{Books}
\label{sec:org5735e00}

There is nothing like a good book to learn from. They are usually rich in content,
but also provide reasonable enough depth to the subject matter to not only learn how
things work, but also why they work the way they do.

\begin{itemize}
\item Think Python: An Introduction to Software Design - Livre d'Allen B. Downey
\item Numerical Python: Scientific Computing and Data Science Applications with Numpy,
SciPy and Matplotlib - Livre de Robert Johansson.
\item Classic Shell Scripting -  Livre de Arnold Robbins, Nelson H F Beebe
\end{itemize}

\subsection{Online resources}
\label{sec:org13eaaf4}

For other, more niche, subject matters, online resources provide the most reasonable
information to learn from.

\begin{itemize}
\item \href{https://slurm.schedmd.com/tutorials.html}{SLURM User Guide}
\end{itemize}

\subsection{Software used in this course}
\label{sec:org1a7cf01}

After we're introduced to the Python programming language, and we're comfortable
creating the most basic scripts, we'll be using a \emph{programming environment} to make
programming easier. When we're ready, we'll discuss both Jupyter notebooks and
the PyCharm IDE.

\begin{itemize}
\item \url{https://jupyter.org/}
\item \url{https://www.jetbrains.com/fr-fr/pycharm/}
\end{itemize}

\section{What will be taught}
\label{sec:org59c8981}

The course will cover a broad spectrum of skills used when programming for scientific
research. This includes the programming and scripting itself (in our case, Python
programming), managing the environment in which we work (i.e. working in a
linux-based environment and managing our projects with version control), and
interacting with the supercomputers to perform intensive computations.

\subsection{Python programming}
\label{sec:orgc6d83a9}
\begin{itemize}
\item Basic syntax
\item Data structures
\item Advanced syntax
\item Modules
\item Anaconda and Pip
\item Different programming development environments
\begin{itemize}
\item PyCharm
\item Jupyter Notebooks
\end{itemize}
\item Numerical computing in python
\begin{itemize}
\item Introduction to numpy
\item Pandas
\item Scipy
\item Visualisation using Matplotlib/seaborn/altair.
\end{itemize}
\end{itemize}
\subsection{GNU/Linux}
\label{sec:org69b314b}
\begin{itemize}
\item Basics of GNU/Linux and terminal
\item Creating bash scripts
\item Using the university proxy
\item Git version control
\end{itemize}
\subsection{Using High Performance/Cloud Computing}
\label{sec:org9da0355}
\begin{itemize}
\item GPU management (CUDA, CuDNN, nvidia-smi)
\item SLURM
\item Google Cloud, AWS
\item Singularity
\end{itemize}
\subsection{Reporting Results}
\label{sec:orga63b3de}
\begin{itemize}
\item Markdown
\item \LaTeX{}
\end{itemize}

\section{A rough timeline}
\label{sec:orgdbab9eb}

I have included below a rough indication of whats going to be taught and when. Of
course, this is subject to change based upon scheduling constraints and rate of
progression.

\begin{center}
\begin{tabular}{rll}
\hline
Week & Topic & Description\\
\hline
1 & Introduction & - Course introduction\\
 &  & - Basic Python programming\\
2 & Python classes & \\
3 & Project management & - Creating/importing modules\\
 &  & - Anaconda/pip\\
4 & Programming environments & - PyCharm\\
 &  & - Jupyter notebooks\\
5 & Numerical computing & - Numpy\\
 &  & - Scipy\\
6 & Numerical computing & - Pandas\\
 &  & - Visualisations\\
7 & Basics of GNU/Linux & - Using the terminal\\
8 & Bash scripting & \\
9 & High performance computing & - SLURM\\
 &  & - Singularity\\
10 & Reporting & - \LaTeX{}\\
 &  & - Markdown\\
\hline
\end{tabular}
\end{center}
\end{document}
