% Created 2021-11-16 mar. 12:03
% Intended LaTeX compiler: pdflatex
\documentclass[10pt]{beamer}
\usepackage[utf8]{inputenc}
\usepackage[T1]{fontenc}
\usepackage{graphicx}
\usepackage{grffile}
\usepackage{longtable}
\usepackage{wrapfig}
\usepackage{rotating}
\usepackage[normalem]{ulem}
\usepackage{amsmath}
\usepackage{textcomp}
\usepackage{amssymb}
\usepackage{capt-of}
\usepackage{hyperref}
\usepackage{minted}
\setlength{\parskip}{5pt}
\newcommand{\footnoteframe}[1]{\footnote[frame]{#1}}
\addtobeamertemplate{footnote}{}{\vspace{2ex}}
\usepackage{tabularx}
\usetheme{Berkeley}
\author{Jay Morgan}
\date{2021-10-01}
\title{Programming Level-up}
\subtitle{Lecture 2 - More advanced Python \& Classes}
\hypersetup{
 pdfauthor={Jay Morgan},
 pdftitle={Programming Level-up},
 pdfkeywords={},
 pdfsubject={},
 pdfcreator={Emacs 27.1 (Org mode 9.4.6)}, 
 pdflang={English}}
\begin{document}

\maketitle
\begin{frame}{Outline}
\tableofcontents
\end{frame}


\section{Proxy}
\label{sec:orgd44dba2}

\subsection{Univ-tln proxy}
\label{sec:orgc6bb8ed}

\begin{frame}[label={sec:org9afa632},fragile]{Setting up a proxy in Linux -- environment variables}
 Environment variables are variables that are set in the Linux environment and are
used to configure some high-level details in Linux.

The command to create/set an environment is:

\begin{verbatim}
export VARIABLE_NAME=""
\end{verbatim}

Exporting a variable in this way will mean \texttt{VARIABLE\_NAME} will be accessible while
you're logged in. Every time you log in you will have to set this variable again.
\end{frame}

\begin{frame}[label={sec:org15ba41f},fragile]{Setting up a proxy in Linux -- univ-tln specific}
 In the université de Toulon, you're required to use the university's proxy server to
access the internet. Therefore, in Linux at least, you will have to tell the system
where the proxy server is with an environment variable.

\begin{minted}[frame=lines,linenos=true,firstnumber=last,fontsize=\footnotesize,xleftmargin=15pt,numbersep=8pt]{bash}
export HTTP_PROXY='<username>:<password>@proxy.univ-tln.fr:3128'
export HTTPS_PROXY='<username>:<password>@proxy.univ-tln.fr:3128'
export FTP_PROXY='<username>:<password>@proxy.univ-tln.fr:3128'
\end{minted}
\end{frame}

\begin{frame}[label={sec:org0e64f4a},fragile]{Setting up a proxy in the .bashrc}
 If you don't wish to set the variable every time log in, you should enter the same
commands into a \texttt{.bashrc} in your home directory.

\begin{minted}[frame=lines,linenos=true,firstnumber=last,fontsize=\footnotesize,xleftmargin=15pt,numbersep=8pt]{bash}
export HTTP_PROXY='...'
export HTTPS_PROXY='...'
export FTP_PROXY='...'
\end{minted}

When you log in, the \texttt{.bashrc} file will be run and these variables will be set for you.
\end{frame}

\section{Advanced syntax}
\label{sec:orgf05c9e6}

\subsection{List comprehensions}
\label{sec:orgd2415c1}

\begin{frame}[label={sec:org4056d08},fragile]{List comprehensions}
 We have seen previously how \texttt{for} loops work. Knowing the syntax of a \texttt{for} loop and
wanting to populate a list with some data, we might be tempted to write:

\begin{minted}[frame=lines,linenos=true,firstnumber=last,fontsize=\footnotesize,xleftmargin=15pt,numbersep=8pt]{python}
x = []
for i in range(3):
    x.append(i)

print(x)
\end{minted}

\begin{verbatim}
Results: 
# => [0, 1, 2]
\end{verbatim}


While this is perfectly valid Python code, Python itself provides 'List
comprehensions' to make this process easier.

\begin{minted}[frame=lines,linenos=true,firstnumber=last,fontsize=\footnotesize,xleftmargin=15pt,numbersep=8pt]{python}
x = [i for i in range(3)]
\end{minted}
\end{frame}

\begin{frame}[label={sec:orge41f586},fragile]{List comprehensions -- syntax}
 The syntax of a list comprehensions is:

\begin{minted}[frame=lines,linenos=true,firstnumber=last,fontsize=\footnotesize,xleftmargin=15pt,numbersep=8pt]{python}
[ <variable> for <variable> in <iterable> ]
\end{minted}

We can also perform similar actions with a dictionary

\begin{minted}[frame=lines,linenos=true,firstnumber=last,fontsize=\footnotesize,xleftmargin=15pt,numbersep=8pt]{python}
[ <key>, <value> for <key>, <value> in <dictionary.items()> ]
\end{minted}
\end{frame}

\begin{frame}[label={sec:org55c49ac},fragile]{List comprehensions -- dictionary}
 Python doesn't restrict us to list comprehensions, but we can do a similar
operation to create a dictionary.

\begin{minted}[frame=lines,linenos=true,firstnumber=last,fontsize=\footnotesize,xleftmargin=15pt,numbersep=8pt]{python}
x = [2, 5, 6]
y = {idx: val for idx, val in enumerate(x)}
print(y)
\end{minted}

\begin{verbatim}
Results: 
# => {0: 2, 1: 5, 2: 6}
\end{verbatim}


Here, every item in \texttt{x} has been associated with its numerical index as a key thanks to
the \texttt{enumerate} function that returns both the index and value at iteration in the for loop.
\end{frame}

\begin{frame}[label={sec:orgce4f0c5},fragile]{List comprehensions -- using \texttt{if}'s}
 Perhaps we only want to optionally perform an action within the list comprehension?
Python allows us to do this with the inline \texttt{if} statement we've seen in the previous lecture.

\begin{minted}[frame=lines,linenos=true,firstnumber=last,fontsize=\footnotesize,xleftmargin=15pt,numbersep=8pt]{python}
x = [i if i < 5 else -1 for i in range(7)]
print(x)
\end{minted}

\begin{verbatim}
Results: 
# => [0, 1, 2, 3, 4, -1, -1]
\end{verbatim}


We add the inline \texttt{<var> if <condition> else <other-var>} before the \texttt{for} loop part of
the comprehension.
\end{frame}

\begin{frame}[label={sec:orgcf7fef6},fragile]{List comprehension -- using \texttt{if}'s}
 There is another type of \texttt{if} statement in a list comprehension, this occurs when we
don't have an \texttt{else}.

\begin{minted}[frame=lines,linenos=true,firstnumber=last,fontsize=\footnotesize,xleftmargin=15pt,numbersep=8pt]{python}
x = [i for i in range(7) if i < 3]
print(x)
\end{minted}

\begin{verbatim}
Results: 
# => [0, 1, 2]
\end{verbatim}


In this example, we're only 'adding' to the list if the condition (\(i < 3\)) is true,
else the element is not included in the resulting list.
\end{frame}

\begin{frame}[label={sec:orga2ee4a3},fragile]{List comprehensions -- multiple \texttt{for}'s}
 If we like, we can also use nested for loops by simply adding another for loop into
the comprehension.

\begin{minted}[frame=lines,linenos=true,firstnumber=last,fontsize=\footnotesize,xleftmargin=15pt,numbersep=8pt]{python}
x = [(i, j) for i in range(2) for j in range(2)]

print(x)
\end{minted}

\begin{verbatim}
Results: 
# => [(0, 0), (0, 1), (1, 0), (1, 1)]
\end{verbatim}


In this example, we're creating a tuple for each element, effectively each
combination of 1 and 0.
\end{frame}

\begin{frame}[label={sec:org5f9c2a0},fragile]{Quick Exercise -- List comprehension}
 \begin{itemize}
\item Using list comprehension, create a list of even numbers from 6 to 20, and assign
this list to the variable named \texttt{even\_numbers}.
\item Create a new variable called \texttt{even\_numbers\_dict}, create a dictionary using the
comprehension syntax. The keys of the dictionary should be the index of each
element in \texttt{even\_numbers}, while the value should be the even number.
\item What is the 5th even number?
\end{itemize}
\end{frame}

\subsection{Exceptions}
\label{sec:org1ebb98f}

\begin{frame}[label={sec:org4302f1e},fragile]{Dealing with Errors}
 When programming, its good to be defensive and handle errors gracefully. For example,
if you're creating a program, that as part of its process, reads from a file, its
possible that this file may not exist at the point the program tries to read it. If
it doesn't exist, the program will crash giving an error such as: \texttt{FileNotfoundError}.

Perhaps this file is non-essential to the operation of the program, and we can
continue without the file. In these cases, we will want to appropriately catch the
error to prevent it from stopping Python.
\end{frame}

\begin{frame}[label={sec:org99543ba},fragile]{Try-catch}
 Try-catches are keywords that introduce a scope where the statements are executed,
and if an error (of a certain type IndexError in this example) occurs, different
statements could be executed.

In this example, we are trying to access an element in a list using an index larger
than the length of the list. This will produce an \texttt{IndexError}. Instead of exiting
Python with an error, however, we can catch the error, and print a string.

\begin{minted}[frame=lines,linenos=true,firstnumber=last,fontsize=\footnotesize,xleftmargin=15pt,numbersep=8pt]{python}
x = [1, 2, 3]

try:
    print(x[3])
except IndexError:
    print("Couldn't access element")
\end{minted}

\begin{verbatim}
Results: 
# => Couldn't access element
\end{verbatim}
\end{frame}

\begin{frame}[label={sec:orgb541120},fragile]{Try-catch -- capturing messages}
 If we wanted to include the original error message in the print statement, we can use
the form:

\begin{verbatim}
except <error> as <variable>
\end{verbatim}

This provides us with an variable containing the original error that we can use later
on in the try-catch form.

\begin{minted}[frame=lines,linenos=true,firstnumber=last,fontsize=\footnotesize,xleftmargin=15pt,numbersep=8pt]{python}
x = [1, 2, 3]

try:
    print(x[3])
except IndexError as e:
    print(f"Couldn't access elements at index beacuse: {e}")
\end{minted}

\begin{verbatim}
Results: 
# => Couldn't access elements at index beacuse: list index out of range
\end{verbatim}
\end{frame}

\begin{frame}[label={sec:org9bec6a4}]{Types of exceptions}
There are numerous types of errors that could occur in a Python. Here are just some
of the most common.

\begin{itemize}
\item IndexError -- Raised when a sequence subscript is out of range.
\item ValueError -- Raised when an operation or function receives an argument that has the right type but an inappropriate value
\item AssertionError -- Raised when an assert statement fails.
\item FileNotFoundError -- Raised when a file or directory is requested but doesn’t exist.
\end{itemize}

The full list of exceptions in Python 3 can be found at: \url{https://docs.python.org/3/library/exceptions.html}
\end{frame}

\begin{frame}[label={sec:orgef905ea},fragile]{Assertions}
 One of the previous errors (\texttt{AssertionError}) occurs when an assert statement
fails. Assert is a keyword provided to test some condition and raise an error if the
condition is false. It typically requires less code than an \texttt{if}-statement that raises
an error, so they might be useful for checking the inputs to functions, for example:

\begin{minted}[frame=lines,linenos=true,firstnumber=last,fontsize=\footnotesize,xleftmargin=15pt,numbersep=8pt]{python}
def my_divide(a, b):
    assert b != 0
    return a / b

my_divide(1, 2)
my_divide(1, 0)
\end{minted}

Here we are checking that the divisor is not a 0, in which case division is not defined.
\end{frame}

\subsection{Working with data}
\label{sec:org9ce25e0}

\begin{frame}[label={sec:orgd44e72a},fragile]{More on lists}
 In a previous lecture, we found that we can add \texttt{.append()} to the end of a variable of
a type list to add an element to the end of the list. Lists have many more methods
associated with them that will be useful when programming in Python.

Lists have a number of other convenient functions\footnoteframe{https://docs.python.org/3/tutorial/datastructures.html}.

Some of these include:

\begin{minted}[frame=lines,linenos=true,firstnumber=last,fontsize=\footnotesize,xleftmargin=15pt,numbersep=8pt]{python}
my_list.insert(0, "dog")  # insert "dog" at index 0
my_list.count(2)          # count the number of times 2 appears
my_list.reverse()         # reverse the list
\end{minted}
\end{frame}

\begin{frame}[label={sec:orgadc7474},fragile]{More on sets -- union}
 Sets, while containing only unique elements, have a number of useful functions to
perform certain set operations. Take for example the union (elements that are in
either sets) of two sets:

\begin{minted}[frame=lines,linenos=true,firstnumber=last,fontsize=\footnotesize,xleftmargin=15pt,numbersep=8pt]{python}
x = set([1, 2, 3, 4, 5])
y = set([5, 2, 6, -1, 10])

print(x.union(y))
\end{minted}

\begin{verbatim}
Results: 
# => {1, 2, 3, 4, 5, 6, 10, -1}
\end{verbatim}


The syntax of using these methods follows:

\begin{verbatim}
<set_1>.function(<set_2>)
\end{verbatim}
\end{frame}

\begin{frame}[label={sec:org52d7459},fragile]{More on sets -- intersection}
 Or the intersection (the elements that are in both) of two sets:

\begin{minted}[frame=lines,linenos=true,firstnumber=last,fontsize=\footnotesize,xleftmargin=15pt,numbersep=8pt]{python}
x = set([1, 2, 3, 4, 5])
y = set([5, 2, 6, -1, 10])

print(x.intersection(y))
\end{minted}

\begin{verbatim}
Results: 
# => {2, 5}
\end{verbatim}
\end{frame}

\begin{frame}[label={sec:org8d122ca},fragile]{More on sets -- set difference}
 And the set difference (the elements that in set 1, but not in set 2):

\begin{minted}[frame=lines,linenos=true,firstnumber=last,fontsize=\footnotesize,xleftmargin=15pt,numbersep=8pt]{python}
x = set([1, 2, 3, 4, 5])
y = set([5, 2, 6, -1, 10])

print(x.difference(y))
\end{minted}

\begin{verbatim}
Results: 
# => {1, 3, 4}
\end{verbatim}
\end{frame}

\begin{frame}[label={sec:org35d5d4d},fragile]{More on set -- subsets}
 We can even return a boolean value if set 1 is a subset of set 2:

\begin{minted}[frame=lines,linenos=true,firstnumber=last,fontsize=\footnotesize,xleftmargin=15pt,numbersep=8pt]{python}
x = set([1, 2, 3, 4, 5])
y = set([5, 2, 6, -1, 10])
z = set([1, 2, 3, 4, 5, 6, 7])

print(x.issubset(y))
print(x.issubset(z))
\end{minted}

\begin{verbatim}
Results: 
# => False
# => True
\end{verbatim}


For a full list of what methods are available with sets, please refer to: \url{https://realpython.com/python-sets/\#operating-on-a-set}
\end{frame}

\begin{frame}[label={sec:org2de9855},fragile]{Better indexing -- slices}
 If we wanted to access an element from a data structure, such as a list, we would use
the \texttt{[ ]} accessor, specifying the index of the element we wish to retrieve (remember
that indexes start at zero!). But what if we ranted to access many elements at once?
Well to accomplish that, we have a slice or a range of indexes (not to be confused
with the \texttt{range} function). A slice is defined as:

\begin{verbatim}
start_index:end_index
\end{verbatim}

where the \texttt{end\_index} is non inclusive -- it doesn't get included in the result. Here
is an example where we have a list of 6 numbers from 0 to 5, and we slice the list from index
0 to 3. Notice how the 3rd index is not included.

\begin{minted}[frame=lines,linenos=true,firstnumber=last,fontsize=\footnotesize,xleftmargin=15pt,numbersep=8pt]{python}
x = [0, 1, 2, 3, 4, 5]
print(x[0:3])
\end{minted}

\begin{verbatim}
Results: 
# => [0, 1, 2]
\end{verbatim}
\end{frame}

\begin{frame}[label={sec:orge06b828},fragile]{Better indexing -- range}
 When we use \texttt{start\_index:end\_index}, the slice increments by 1 from \texttt{start\_index} to
\texttt{end\_index}. If we wanted to increment by a different amount we can use the slicing
form:

\begin{verbatim}
start_index:end_index:step
\end{verbatim}

Here is an example where we step the indexes by 2:

\begin{minted}[frame=lines,linenos=true,firstnumber=last,fontsize=\footnotesize,xleftmargin=15pt,numbersep=8pt]{python}
x = list(range(100))
print(x[10:15:2])
\end{minted}

\begin{verbatim}
Results: 
# => [10, 12, 14]
\end{verbatim}
\end{frame}

\begin{frame}[label={sec:org7a61cc0},fragile]{Better indexing -- reverse}
 One strange fact about the step is that if we specify a negative number for the step,
Python will work backwards, and effectively reverse the list.

\begin{minted}[frame=lines,linenos=true,firstnumber=last,fontsize=\footnotesize,xleftmargin=15pt,numbersep=8pt]{python}
x = list(range(5))

print(x[::-1])
\end{minted}

\begin{verbatim}
Results: 
# => [4, 3, 2, 1, 0]
\end{verbatim}
\end{frame}

\begin{frame}[label={sec:org50df1bd},fragile]{Better indexing -- range}
 In a previous example, I created a slice like \texttt{0:3}. This was a little wasteful as we
can write slightly less code. If we write \texttt{:end\_index}, Python assumes and creates a
slice from the first index (0) to the \texttt{end\_index}. If we write \texttt{start\_index:}, Python
assumes and creates a slice from \texttt{start\_index} to the end of the list.

\begin{minted}[frame=lines,linenos=true,firstnumber=last,fontsize=\footnotesize,xleftmargin=15pt,numbersep=8pt]{python}
x = list(range(100))

print(x[:10])
print(x[90:])
\end{minted}

\begin{verbatim}
Results: 
# => [0, 1, 2, 3, 4, 5, 6, 7, 8, 9]
# => [90, 91, 92, 93, 94, 95, 96, 97, 98, 99]
\end{verbatim}
\end{frame}

\begin{frame}[label={sec:orgcf41742},fragile]{Better indexing -- backwards}
 Finally, we also work backwards from the end of list. If we use a negative number,
such as -1, we are telling Python, take the elements from the end of the list. -1 is
the final index, and numbers lower than -1 work further backwards through the list.

\begin{minted}[frame=lines,linenos=true,firstnumber=last,fontsize=\footnotesize,xleftmargin=15pt,numbersep=8pt]{python}
x = list(range(100))

print(x[-1])
print(x[-2])
\end{minted}

\begin{verbatim}
Results: 
# => 99
# => 98
\end{verbatim}
\end{frame}

\begin{frame}[label={sec:orgc57fd66},fragile]{Better indexing --backwards}
 Slicing with negative indexes, also works. Here we are creating a slice from the end
of the list - 10, to the last (but not including) index.

\begin{minted}[frame=lines,linenos=true,firstnumber=last,fontsize=\footnotesize,xleftmargin=15pt,numbersep=8pt]{python}
x = list(range(100))

print(x[-10:-1])
\end{minted}

\begin{verbatim}
Results: 
# => [90, 91, 92, 93, 94, 95, 96, 97, 98]
\end{verbatim}
\end{frame}

\begin{frame}[label={sec:org8aaae30}]{Quick Exercise -- Slicing}
\begin{itemize}
\item Create a list of elements from 0 to 100, of every 3rd number (e.g. use a range
with a step).
\item First, slice the first 5 indexes.
\item Second, get the last 10 indexes.
\item Third, get the 50th to 55th (inclusive) indexes.
\item \alert{Challenge} get the last 10 indexes, but only using positive indexes up to 10.
\end{itemize}
\end{frame}

\subsection{Working with strings}
\label{sec:org8e6ac0a}

\begin{frame}[label={sec:org6ee7f76},fragile]{Formatting strings}
 In many previous examples when we've printed strings, we've done something like:

\begin{minted}[frame=lines,linenos=true,firstnumber=last,fontsize=\footnotesize,xleftmargin=15pt,numbersep=8pt]{python}
age = 35

print("The value of age is", age)
\end{minted}

\begin{verbatim}
Results: 
# => The value of age is 35
\end{verbatim}


While this works in this small context, it can get pretty cumbersome if we have many
variables we want to print, and we also want to change how they are displayed when
they are printed.

We're going to take a look now at much better ways of printing.
\end{frame}

\begin{frame}[label={sec:orgdcb6b85},fragile]{Better ways of printing strings - \%}
 The first method is using \texttt{\%}. When we print, we first construct a string with special
delimiters, such as \texttt{\%s} that denotes a string, and \texttt{\%d} that denotes a number. This is
telling Python where we want the values to be placed in the string.

Once we've created the string, we need to specify the data, which we do with \texttt{\%
(...)}. Like, for example:

\begin{minted}[frame=lines,linenos=true,firstnumber=last,fontsize=\footnotesize,xleftmargin=15pt,numbersep=8pt]{python}
age = 35
name = "John"

print("%d years old" % age)  # no tuple for one variable
print("%s is %d years old" % (name, age)) 
\end{minted}

\begin{verbatim}
Results: 
# => 35 years old
# => John is 35 years old
\end{verbatim}


Here we are specifying the a string \texttt{\%s} and number \texttt{\%d}, and then giving the variables
that correspond with that data type.
\end{frame}

\begin{frame}[label={sec:org946ad0a},fragile]{Better ways of printing strings -- data specifiers}
 The special delimiters correspond with a data type. Here are some of the most common:

\begin{itemize}
\item \texttt{\%s} -- For strings
\item \texttt{\%d} -- For numbers
\item \texttt{\%f} -- For floating point numbers.
\end{itemize}

There are others such as \texttt{\%x} that prints the hexadecimal representation, but these are
less common. You can find the full list at: \url{https://docs.python.org/3/library/stdtypes.html\#old-string-formatting}
\end{frame}

\begin{frame}[label={sec:org99f29a4},fragile]{Better ways of printing strings -- floating points}
 When using these delimiters, we can add modifiers to how they format and display the
value. Take a very common example, where we have a floating point value, and, when
printing it, we only want to print to 3 decimal places. To accomplish this, we again
use \texttt{\%f} but add a \texttt{.3} to between the \texttt{\%} and \texttt{f}. In this example, we are printing \(\pi\) to 3
decimal places.

\begin{minted}[frame=lines,linenos=true,firstnumber=last,fontsize=\footnotesize,xleftmargin=15pt,numbersep=8pt]{python}
print("Pi to 3 digits is: %.3f" % 3.1415926535)
\end{minted}

\begin{verbatim}
Results: 
# => Pi to 3 digits is: 3.142
\end{verbatim}
\end{frame}

\begin{frame}[label={sec:org2fc0171},fragile]{Better ways of printing strings -- floating points}
 In the previous example, we used \texttt{.3} to specify 3 decimal places. If we put a number
before the decimal, like \texttt{10.3} we are telling Python \emph{make this float occupy 10 spaces
and this float should have 3 decimal places printed}.  When it gets printed, you will
notice that it shifts to the right, it gets padded by space. If we use a negative
number in front of the decimal place, we are telling python to shift it to the left.

\begin{minted}[frame=lines,linenos=true,firstnumber=last,fontsize=\footnotesize,xleftmargin=15pt,numbersep=8pt]{python}
print("Pi to 3 digits is: %10.3f" % 3.1415926535)
print("Pi to 3 digits is: %-10.3f" % 3.1415926535)
\end{minted}

\begin{verbatim}
Results: 
# => Pi to 3 digits is:      3.142
# => Pi to 3 digits is: 3.142
\end{verbatim}
\end{frame}

\begin{frame}[label={sec:org75dc67c},fragile]{Quick Exercise -- printing with \texttt{\%}}
 \begin{itemize}
\item Creating a dictionary containing the following information:
\end{itemize}

\begin{center}
\begin{tabular}{lr}
Key & Value\\
\hline
name & Jane\\
age & 35\\
lon & -3.52352\\
lat & 2.25222\\
\end{tabular}
\end{center}

\begin{itemize}
\item Print (using the \texttt{\%} operator) the values of this dictionary so that the result looks
like: "Jane (located at -3.5, 2.2) is 35 years old"
\end{itemize}
\end{frame}

\begin{frame}[label={sec:org598c8ae},fragile]{Better ways of printing strings -- \texttt{.format()}}
 Another way of performing 'string interpolation' where the values associated with
variables are printed with strings is accomplished using the \texttt{.format()} method.

To use this method, create a string with \texttt{\{\}} delimiters, and after the string, call
the \texttt{.format()} method, where the arguments to this method are the values you want to
include in the string. The number of values passed to \texttt{.format()} should be the same as
the number of \texttt{\{\}} in the string.

\begin{minted}[frame=lines,linenos=true,firstnumber=last,fontsize=\footnotesize,xleftmargin=15pt,numbersep=8pt]{python}
name = "Jane"
age = 35

print("{} is {} years old".format(name, age))
\end{minted}

\begin{verbatim}
Results: 
# => Jane is 35 years old
\end{verbatim}
\end{frame}

\begin{frame}[label={sec:orgf926d1f},fragile]{Better ways of printing strings -- \texttt{.format()}}
 To be more explicit and clear with which values go where in the string, we can name
them by putting some same into the \texttt{\{\}} tokens. When we call the \texttt{.format()} function, we
then use the same name as named parameters.

\begin{minted}[frame=lines,linenos=true,firstnumber=last,fontsize=\footnotesize,xleftmargin=15pt,numbersep=8pt]{python}
name = "Jane"
age = 35

print("{the_name} is {the_age} years old".format(the_name=name,
						 the_age=age))
\end{minted}

\begin{verbatim}
Results: 
# => Jane is 35 years old
\end{verbatim}
\end{frame}

\begin{frame}[label={sec:org31ee4d4},fragile]{Better ways of printing strings -- alignment}
 \texttt{.format()} allows us to some quite complicated things with the display of
strings. Take this for example where we are setting the alignment of the values.

The syntax of formatting strings can be a language of it's own right! So we won't go
too deep into it here. However, you can find all you need to know about formatting
here: \url{https://docs.python.org/3/library/string.html\#format-string-syntax}

\begin{minted}[frame=lines,linenos=true,firstnumber=last,fontsize=\footnotesize,xleftmargin=15pt,numbersep=8pt]{python}
print("|{:<10}|{:^10}|{:>10}|".format('all','dogs','bark'))
print("-" * 34)
\end{minted}

\begin{verbatim}
Results: 
# => |all       |   dogs   |      bark|
# => ----------------------------------
\end{verbatim}
\end{frame}

\begin{frame}[label={sec:org160bdfc},fragile]{Better ways of printing strings -- \texttt{f-strings}}
 The final method of formatting strings is a newcomer within the language, it is the
so-called \texttt{f-string}. Where a \texttt{f} character is prefixed to the beginning of the string
you're creating. \texttt{f-string}'s allow you to use Python syntax within the string (again
delimited by \texttt{\{\}}.

Take this for example where we are referencing the variables \texttt{name} and \texttt{age} directly.

\begin{minted}[frame=lines,linenos=true,firstnumber=last,fontsize=\footnotesize,xleftmargin=15pt,numbersep=8pt]{python}
name = "Jane"
age = 35

print(f"{name} is {age} years old")
\end{minted}

\begin{verbatim}
Results: 
# => Jane is 35 years old
\end{verbatim}
\end{frame}

\begin{frame}[label={sec:org4e7453a},fragile]{Better ways of printing strings -- \texttt{f-strings}}
 \texttt{f-string}'s allow you to execute Python code within the string. Here we are accessing
the value from the dictionary by specifying the key within the string itself! It
certainly makes it a lot easier, especially if we only need to access the values for
the string itself.

\begin{minted}[frame=lines,linenos=true,firstnumber=last,fontsize=\footnotesize,xleftmargin=15pt,numbersep=8pt]{python}
contact_info = {"name": "Jane", "age": 35}

print(f"{contact_info['name']} is {contact_info['age']} years old")
\end{minted}

\begin{verbatim}
Results: 
# => Jane is 35 years old
\end{verbatim}


\url{https://pyformat.info/}
\end{frame}

\begin{frame}[label={sec:org3469f16},fragile]{Better ways of printing strings -- \texttt{f-string}}
 We can still format the values when using \texttt{f-string}. The method is similar to those
using the \texttt{\%f} specifiers.

\begin{minted}[frame=lines,linenos=true,firstnumber=last,fontsize=\footnotesize,xleftmargin=15pt,numbersep=8pt]{python}
pi = 3.1415926535
print(f"Pi is {pi:.3f} to 3 decimal places")
\end{minted}

\begin{verbatim}
Results: 
# => Pi is 3.142 to 3 decimal places
\end{verbatim}


Many more examples can be found at: \url{https://zetcode.com/python/fstring/}
\end{frame}

\begin{frame}[label={sec:orgd1df163},fragile]{Quick Exercise -- printing with \texttt{f-string}}
 \begin{itemize}
\item Creating a dictionary containing the following information:
\end{itemize}

\begin{center}
\begin{tabular}{lr}
Key & Value\\
\hline
name & Jane\\
age & 35\\
lon & -3.52352\\
lat & 2.25222\\
\end{tabular}
\end{center}

\begin{itemize}
\item Print (using an \texttt{f-string}) the values of this dictionary so that the result looks
like: "Jane (located at -3.5, 2.2) is 35 years old"
\end{itemize}
\end{frame}

\begin{frame}[label={sec:org46a2e6c},fragile]{Operations on strings -- splitting}
 Apart from formatting, there are plenty more operations we can perform on strings. We
are going to highlight some of the most common here.

The first we're going to look at is splitting a string by a delimiter character using
the \texttt{.split()} method. If we don't pass any argument to the \texttt{.split()} method, then by
default, it will split by spaces. However, we can change this by specifying the
delimiter.

\begin{minted}[frame=lines,linenos=true,firstnumber=last,fontsize=\footnotesize,xleftmargin=15pt,numbersep=8pt]{python}
my_string = "This is a sentence, where each word is separated by a space"

print(my_string.split())
print(my_string.split(","))
\end{minted}

\begin{verbatim}
Results: 
# => ['This', 'is', 'a', 'sentence,', 'where', 'each', 'word', 'is', 'separated', 'by', 'a', 'space']
# => ['This is a sentence', ' where each word is separated by a space']
\end{verbatim}
\end{frame}

\begin{frame}[label={sec:org61d6765},fragile]{Operations on strings -- joining}
 As \texttt{.split()} splits a single string into a list, \texttt{.join()} joins a list of strings into
a single string. To use \texttt{.join()}, we first create a string of the delimiter we want to
use to join the list of strings by. In this example we're going to use \texttt{"-"}. Then we
call the \texttt{.join()} method, passing the list as an argument.

The result is a single string using the delimiter to separate the items of the list.

\begin{minted}[frame=lines,linenos=true,firstnumber=last,fontsize=\footnotesize,xleftmargin=15pt,numbersep=8pt]{python}
x = ['This', 'is', 'a', 'sentence,', 'where', 'each', 'word', 'is', 'separated', 'by', 'a', 'space']

print("-".join(x))
\end{minted}

\begin{verbatim}
Results: 
# => This-is-a-sentence,-where-each-word-is-separated-by-a-space
\end{verbatim}
\end{frame}

\begin{frame}[label={sec:orgdf44a6a},fragile]{Operations on strings -- changing case}
 Other common operations on strings involve change the case. For example:

\begin{itemize}
\item Make the entire string uppercase or lowercase
\item Making the string title case (every where starts with a capital letter).
\item Stripping the string by removing any empty spaces either side of the string.
\end{itemize}

\alert{Note} we can chain many methods together by doing \texttt{.method\_1().method\_2()}, but only if they
return string. If they return \texttt{None}, then chaining will not work.

\begin{minted}[frame=lines,linenos=true,firstnumber=last,fontsize=\footnotesize,xleftmargin=15pt,numbersep=8pt]{python}
x = "    this String Can change case"

print(x.upper())
print(x.lower())
print(x.title())
print(x.strip())
print(x.strip().title())
\end{minted}

\begin{verbatim}
Results: 
# =>     THIS STRING CAN CHANGE CASE
# =>     this string can change case
# =>     This String Can Change Case
# => this String Can change case
# => This String Can Change Case
\end{verbatim}
\end{frame}

\begin{frame}[label={sec:org2edea42},fragile]{Operations on strings -- replacing strings}
 To replace a substring, we use the \texttt{.replace()} method. The first argument is the old
string you want to replace. The second argument is what you want to replace it with.

\begin{minted}[frame=lines,linenos=true,firstnumber=last,fontsize=\footnotesize,xleftmargin=15pt,numbersep=8pt]{python}
x = "This is a string that contains some text"

print(x.replace("contains some", "definitely contains some"))
\end{minted}

\begin{verbatim}
Results: 
# => This is a string that definitely contains some text
\end{verbatim}
\end{frame}

\begin{frame}[label={sec:org8abc3c6},fragile]{Operations on strings -- does it contain a substring?}
 We can check if a string exists within another string using the \texttt{in} keyword. This
returns a Boolean value, so we can use it as a condition to an \texttt{if} statement.

\begin{minted}[frame=lines,linenos=true,firstnumber=last,fontsize=\footnotesize,xleftmargin=15pt,numbersep=8pt]{python}
x = "This is a string that contains some text"

if "text" in x:
    print("It exists")
\end{minted}

\begin{verbatim}
Results: 
# => It exists
\end{verbatim}
\end{frame}

\section{OOP}
\label{sec:org8b37cdc}
\subsection{Classes}
\label{sec:orgd3faa2b}

\begin{frame}[label={sec:org2bd2349}]{Introduction to classes}
A class is some representation (can be abstract) of an object. Classes can be used to
create some kind of structure that can be manipulated and changed, just like the ways
you've seen with lists, dictionaries, etc.

Classes allow us to perform Object-oriented Programming (OOP), where we represent
concepts by classes.

But to properly understand how classes work, and why we would want to use them, we
should take a look at some examples.
\end{frame}

\begin{frame}[label={sec:org5b77321},fragile]{Basic syntax}
 We're going to start off with the very basic syntax, and build up some more complex
classes.

To create a class, we use the \texttt{class} keyword, and give our new class a name. This
introduces a new scope in Python, the scope of the class.

Typically, the first thing we shall see in the class is the \texttt{\_\_init\_\_} function.

\begin{minted}[frame=lines,linenos=true,firstnumber=last,fontsize=\footnotesize,xleftmargin=15pt,numbersep=8pt]{python}
class <name_of_class>:
    def __init__(self, args*):
	<body>
\end{minted}
\end{frame}

\begin{frame}[label={sec:orgc9ff9ac},fragile]{Init method}
 The \texttt{\_\_init\_\_} function is a function that gets called automatically as soon as a class
is made. This init function can take many arguments, but must always start with a
\texttt{self}.

In this example, we are creating a class that represents an x, y coordinate. We've
called this class \texttt{Coordinate}, and we've defined our init function to take an x and y
values when the class is being created.

\alert{Note} its more typical to use titlecase when specifying the class name. So when
reading code its easy to see when you're creating a class versus calling a
function. You should use this style.

\begin{minted}[frame=lines,linenos=true,firstnumber=last,fontsize=\footnotesize,xleftmargin=15pt,numbersep=8pt]{python}
class Coordinate:
    def __init__(self, x, y):
	self.x = x
	self.y = y
\end{minted}
\end{frame}

\begin{frame}[label={sec:orge903ad1},fragile]{Instantiating}
 To create an \emph{instance} of this class, call the name of the class as you would a
function, and pass any parameters you've defined in the init function.

In this example, we are creating a new vector using \texttt{Vector(...)} and we're passing the
x and y coordinate.

\begin{minted}[frame=lines,linenos=true,firstnumber=last,fontsize=\footnotesize,xleftmargin=15pt,numbersep=8pt]{python}
class Vector:
    def __init__(self, x, y):
	self.x = x
	self.y = y


point_1 = Vector(5, 2)
\end{minted}
\end{frame}

\begin{frame}[label={sec:org991d30a},fragile]{Class variables}
 In the previous example, we've been creating a class variables by using
\texttt{self.<variable\_name>}. This is telling Python \emph{this class should have a variable of
this name}.

It allows then to reference the variable when working with the class.

\begin{minted}[frame=lines,linenos=true,firstnumber=last,fontsize=\footnotesize,xleftmargin=15pt,numbersep=8pt]{python}
class Vector:
    def __init__(self, x, y):
	self.x = x
	self.y = y
	self.length = self.x + self.y

point_1 = Vector(5, 2)
print(point_1.x)
print(point_1.y)
print(point_1.length)
\end{minted}

\begin{verbatim}
Results: 
# => 5
# => 2
# => 7
\end{verbatim}
\end{frame}

\begin{frame}[label={sec:orgec1535d},fragile]{Class Methods}
 A class can have many methods associated with it. To create a new method, we create a
function within the scope of the class, remember that the first parameter of the
function should be \texttt{self}.

Even in these functions, we can refer to our \texttt{self.x} and \texttt{self.y} within this new
function.

You'll notice that to call this function, we using the \texttt{.length()} method similar to
how we've worked with strings/lists/etc. This is because in Python, everything is an
object!

\begin{minted}[frame=lines,linenos=true,firstnumber=last,fontsize=\footnotesize,xleftmargin=15pt,numbersep=8pt]{python}
class Vector:
    def __init__(self, x, y):
	self.x = x
	self.y = y

    def length(self):
	return self.x + self.y


my_point = Vector(2, 5)
print(my_point.length())
\end{minted}

\begin{verbatim}
Results: 
# => 7
\end{verbatim}
\end{frame}

\begin{frame}[label={sec:org15e374f},fragile]{dunder-methods}
 While we could, for example, create a function called \texttt{.print()}, sometimes we would
like to use the in built functions like \texttt{print()}. When creating a class, there is a
set of \emph{dunder-methods} (double-under to reference the two '\texttt{\_\_}' characters either side
of the function name).

One of these dunder-methods is \texttt{\_\_repr\_\_}, which allows us to specify how the object
looks when its printed.
\end{frame}

\begin{frame}[label={sec:orgf922e11},fragile]{dunder-methods}
 \begin{minted}[frame=lines,linenos=true,firstnumber=last,fontsize=\footnotesize,xleftmargin=15pt,numbersep=8pt]{python}
class OldVector:
    def __init__(self, x, y):
	self.x = x
	self.y = y

print(OldVector(2, 5))

class Vector:
    def __init__(self, x, y):
	self.x = x
	self.y = y

    def __repr__(self):
	return f"Vector({self.x}, {self.y})"

print(Vector(2, 5))
\end{minted}

\begin{verbatim}
Results: 
# => <__main__.OldVector object at 0x7f658721e250>
# => Vector(2, 5)
\end{verbatim}
\end{frame}

\begin{frame}[label={sec:org983105c},fragile]{dunder-methods}
 There are many more dunder-methods you should know when creating classes. We shall go through:

\begin{itemize}
\item \texttt{\_\_len\_\_} -- specify how the length of the class should be computed.
\item \texttt{\_\_getitem\_\_} -- how to index over the class
\item \texttt{\_\_call\_\_} -- how to use the class like a function
\item \texttt{\_\_iter\_\_} -- what to do when iteration starts
\item \texttt{\_\_next\_\_} -- what to do at the next step of the iteration
\end{itemize}
\end{frame}

\begin{frame}[label={sec:org6b76a12},fragile]{\texttt{\_\_len\_\_}}
 The \texttt{\_\_len\_\_} function allows us to specify how the \texttt{len()} function acts on the
class. Take this hypothetical dataset. We create a \texttt{\_\_len\_\_} function that returns the
length of the unique elements in the dataset.

\begin{minted}[frame=lines,linenos=true,firstnumber=last,fontsize=\footnotesize,xleftmargin=15pt,numbersep=8pt]{python}
class Dataset:
    def __init__(self, data):
	self.data = data

    def __len__(self):
	"""Return the length of unique elements"""
	return len(set(self.data))

data = Dataset([1, 2, 3, 3, 3, 5, 1])
print(len(data))
\end{minted}

\begin{verbatim}
Results: 
# => 4
\end{verbatim}
\end{frame}

\begin{frame}[label={sec:org4f41238},fragile]{\texttt{\_\_getitem\_\_}}
 Next \texttt{\_\_getitem\_\_} allows us to index over a class. This new function must include \texttt{self}
and a variable to pass the index. Here I've used \texttt{idx}. In this function I am simply
indexing on the on the classes \texttt{self.data}.

\begin{minted}[frame=lines,linenos=true,firstnumber=last,fontsize=\footnotesize,xleftmargin=15pt,numbersep=8pt]{python}
class Dataset:
    def __init__(self, data):
	self.data = data

    def __getitem__(self, idx):
	return self.data[idx]

data = Dataset([1, 2, 3, 3, 3, 5, 1])
print(data[2])
\end{minted}

\begin{verbatim}
Results: 
# => 3
\end{verbatim}
\end{frame}

\begin{frame}[label={sec:orga189da9},fragile]{\texttt{\_\_call\_\_}}
 In a small number of cases, it is nice to use the class just like a function. This is
what \texttt{\_\_call\_\_} allows us to do. In this function we specify what should happen when
class is 'called' like a function. In this simple example, we are creating a function
that prints the type of food being used as a parameter to the function.

\begin{minted}[frame=lines,linenos=true,firstnumber=last,fontsize=\footnotesize,xleftmargin=15pt,numbersep=8pt]{python}
class Jaguar:
    def __call__(self, food):
	print(f"The jaguar eats the {food}.")

food = "apple"
animal = Jaguar()

animal(food)
\end{minted}

\begin{verbatim}
Results: 
# => The jaguar eats the apple.
\end{verbatim}
\end{frame}

\begin{frame}[label={sec:orgcc497dd},fragile]{\texttt{\_\_iter\_\_} and \texttt{\_\_next\_\_}}
 \texttt{\_\_iter\_\_} and \texttt{\_\_next\_\_} allow us to make our class iterable, i.e. we can use it in a
\texttt{for} loop for example.

The \texttt{\_\_iter\_\_} function should define what happens when we start the iteration, and
\texttt{\_\_next\_\_} defines what happens at every step of the iteration.

Let's take a look at an example where we have an iterable set of prime numbers.
\end{frame}

\begin{frame}[label={sec:org2740e93},fragile]{\texttt{\_\_iter\_\_} and \texttt{\_\_next\_\_}}
 \begin{minted}[frame=lines,linenos=true,firstnumber=last,fontsize=\footnotesize,xleftmargin=15pt,numbersep=8pt]{python}
class Primes:
    def __init__(self):
	self.primes = [2, 3, 5, 7, 11]

    def __iter__(self):
	self.idx = 0
	return self

    def __len__(self):
	return len(self.primes)

    def __next__(self):
	if self.idx < len(self):
	    item = self.primes[self.idx]
	    self.idx += 1
	    return item
	else:
	    raise StopIteration
\end{minted}
\end{frame}

\begin{frame}[label={sec:org4e17a9b},fragile]{\texttt{\_\_iter\_\_} and \texttt{\_\_next\_\_}}
 And now we can iterate over this class

\begin{minted}[frame=lines,linenos=true,firstnumber=last,fontsize=\footnotesize,xleftmargin=15pt,numbersep=8pt]{python}
prime_numbers = Primes()

for prime_number in prime_numbers:
    print(prime_number)
\end{minted}

\begin{verbatim}
Results: 
# => 2
# => 3
# => 5
# => 7
# => 11
\end{verbatim}
\end{frame}

\begin{frame}[label={sec:org7eba47b},fragile]{Inheritance}
 One special thing about OOP is that its normally designed to provide inheritance --
this is true in Python. Inheritance is where you have a base class, and other classes
inherit from this base class. This means that the class that inherits from the base
class has access to the same methods and class variables. In some cases, it can
override some of these features.

Let's take a look an example.

\begin{minted}[frame=lines,linenos=true,firstnumber=last,fontsize=\footnotesize,xleftmargin=15pt,numbersep=8pt]{python}
class Animal:
    def growl(self):
	print("The animal growls")

    def walk(self):
	raise NotImplementError
\end{minted}

Here we have created a simple class called Animal, that has two functions, one of
which will raise an error if its called.
\end{frame}

\begin{frame}[label={sec:orgbe1540f},fragile]{Inheritance}
 We can inherit from this Animal class by placing our base class in \texttt{()} after the new
class name.

Here we are creating two classes, Tiger and Duck. Both of these new classes inherit
from Animal. Also, both of these classes are overriding the walk functions. But they
are not creating a growl method themselves.

\begin{minted}[frame=lines,linenos=true,firstnumber=last,fontsize=\footnotesize,xleftmargin=15pt,numbersep=8pt]{python}
class Tiger(Animal):
    def walk(self):
	print("The Tiger walks through the jungle")

class Duck(Animal):
    def walk(self):
	print("The Duck walks through the jungle")
\end{minted}
\end{frame}

\begin{frame}[label={sec:org372ba81},fragile]{Inheritance}
 Look at what happens when we create instances of these classes, and call the
functions. First we see that the correct method has been called. I.e. for the duck
class, the correct \texttt{walk} method was called.

\begin{minted}[frame=lines,linenos=true,firstnumber=last,fontsize=\footnotesize,xleftmargin=15pt,numbersep=8pt]{python}
first_animal = Tiger()
second_animal = Duck()

first_animal.walk()
second_animal.walk()
\end{minted}

\begin{verbatim}
Results: 
# => The Tiger walks through the jungle
# => The Duck walks through the jungle
\end{verbatim}
\end{frame}

\begin{frame}[label={sec:org2fabbe0},fragile]{Inheritance}
 But what happens if we call the \texttt{.growl()} method?

\begin{minted}[frame=lines,linenos=true,firstnumber=last,fontsize=\footnotesize,xleftmargin=15pt,numbersep=8pt]{python}
first_animal.growl()
second_animal.growl()
\end{minted}

\begin{verbatim}
Results: 
# => The animal growls
# => The animal growls
\end{verbatim}


We see that it still works. Even though both Duck and Tiger didn't create a \texttt{.growl()}
method, it inherited it from the base class Animal. This works for class methods and
class variables.
\end{frame}

\section{Exercise}
\label{sec:org7c01ad5}

\subsection{Exercise}
\label{sec:orgce09ccb}

\begin{frame}[label={sec:org517c442},fragile]{An object based library system}
 We're going to improve on our library system from last lecture. Instead of a
\texttt{functional} style of code, we're going to use a OOP paradigm to create our solution.

Like last time, we're going to create our solution one step at a time.

First, we need to create our class called \texttt{Database}. This database is going to take an
optional parameter in its init function -- the data. If the user specifies data
(represented as a list of dictionaries like last time), then the class will populate
a class variable called data, else this class variable will be set to an empty list.

Summary:
\begin{itemize}
\item Create a class called \texttt{Database}.
\item When creating an instance of \texttt{Database}, the user can optionally specify a list of
dictionaries to initialise the class variable \texttt{data} with. If no data is provided,
this class variable will be initialised to an empty list.
\end{itemize}
\end{frame}

\begin{frame}[label={sec:org3234d7f},fragile]{Adding data}
 We will want to include a function to add data to our database.

Create a class method called \texttt{add}, that takes three arguments (in addition to \texttt{self} of
course), the title, the author, and the release date.

This add function adds the new book entry to the end of \texttt{data}. Populate this database
with the following information.

\begin{center}
\begin{tabularx}{\textwidth}{XXX}
Title & Author & Release Date\\
\hline
Moby Dick & Herman Melville & 1851\\
A Study in Scarlet & Sir Arthur Conan Doyle & 1887\\
Frankenstein & Mary Shelley & 1818\\
Hitchhikers Guide to the Galaxy & Douglas Adams & 1879\\
\end{tabularx}
\end{center}
\end{frame}

\begin{frame}[label={sec:org21bf358},fragile]{Locating a book}
 Create a class method called locate by tile that takes the title of the book to look
up, and returns the dictionary of all books that have this title. Unlike last time,
we don't need to pass the \texttt{data} as an argument, as its contained within the class.
\end{frame}

\begin{frame}[label={sec:org2449981},fragile]{Updating our database}
 Create a class method called \texttt{update} that takes 4 arguments:, 1) the key of the value
we want to update 2) the value we want to update it to 3) the key we want to check
to find out if we have the correct book and 4) the value of the key to check if we
have the correct book.

\begin{minted}[frame=lines,linenos=true,firstnumber=last,fontsize=\footnotesize,xleftmargin=15pt,numbersep=8pt]{python}
db.update(key="release year", value=1979, where_key="title",
	  where_value="Hitchhikers Guide to the Galaxy")
\end{minted}

Use this to fix the release data of the Hitchhiker's book.
\end{frame}

\begin{frame}[label={sec:org01ac451},fragile]{Printed representation}
 Using the \texttt{\_\_str\_\_} dunder-method (this is similar to \texttt{\_\_repr\_\_} as we saw before),
create a function that prints out a formatted representation of the entire database
as a string. Some of the output should look like:

\begin{minted}[frame=lines,linenos=true,firstnumber=last,fontsize=\footnotesize,xleftmargin=15pt,numbersep=8pt]{python}
Library System
--------------

Entry 1:
- Name: Moby Dick
- Author: Herman Melville
- Release Date: 1851
...
\end{minted}
\end{frame}

\begin{frame}[label={sec:org69ca952},fragile]{Extending our OOP usage}
 So far we've used a list of dictionaries. One issue with this is that there is no
constraints on the keys we can use. This will certainly create problems if certain
keys are missing.

Instead of using dictionaries. We can create another class called \texttt{Book} that will take
three arguments when it is initialised: \texttt{name}, \texttt{author}, and \texttt{release\_date}. The init
function should initialise three class variables to save this information.

Modify the database to, instead of working with a list of dictionaries, work with a
list of Book objects.
\end{frame}

\begin{frame}[label={sec:orga97dc50},fragile]{Printed representation -- challenge.}
 Improve upon the printed representation of the last exercise but instead of bulleted
lists, use formatted tables using \texttt{f-string} formatting
(\url{https://zetcode.com/python/fstring/}).

The output should look like this:

\begin{minted}[frame=lines,linenos=true,firstnumber=last,fontsize=\footnotesize,xleftmargin=15pt,numbersep=8pt]{python}
Library System
--------------

| Name           | Author           | Release Data |
|----------------|------------------|--------------|
| Moby Dick      | Herman Melville  |         1851 |
...
\end{minted}

Notice how Release date is right justified, while Name and Author are left justified.
\end{frame}
\end{document}
